% Opcje klasy 'iithesis' opisane sa w komentarzach w pliku klasy. Za ich pomoca
% ustawia sie przede wszystkim jezyk oraz rodzaj (lic/inz/mgr) pracy.
\documentclass[english, shortabstract]{iithesis}

\usepackage[utf8]{inputenc}

%%%%% DANE DO STRONY TYTUŁOWEJ
% Niezaleznie od jezyka pracy wybranego w opcjach klasy, tytul i streszczenie
% pracy nalezy podac zarowno w jezyku polskim, jak i angielskim.
% Pamietaj o madrym (zgodnym z logicznym rozbiorem zdania oraz estetyka) recznym
% zlamaniu wierszy w temacie pracy, zwlaszcza tego w jezyku pracy. Uzyj do tego
% polecenia \fmlinebreak.
\polishtitle    {Wymagający złamania wierszy\fmlinebreak tytuł pracy w~języku polskim}
\englishtitle   {Resolution for Forward-Guarded Fragment}
\polishabstract {\ldots}
\englishabstract{\ldots}
% w pracach wielu autorow nazwiska mozna oddzielic poleceniem \and
\author         {Karol Ochman-Milarski}
% w przypadku kilku promotorow, lub koniecznosci podania ich afiliacji, linie
% w ponizszym poleceniu mozna zlamac poleceniem \fmlinebreak
\advisor        {dr Jan Kowalski}
%\date          {}                     % Data zlozenia pracy
% Dane do oswiadczenia o autorskim wykonaniu
%\transcriptnum {}                     % Numer indeksu
%\advisorgen    {dr. Jana Kowalskiego} % Nazwisko promotora w dopelniaczu
%%%%%

%%%%% WLASNE DODATKOWE PAKIETY
%
% \usepackage{graphicx,listings,amsmath,amssymb,amsthm,amsfonts,tikz}
\usepackage{graphicx,listings, amsmath, amssymb, amsthm, amsfonts}
\usepackage{algorithm}
\usepackage[noend]{algpseudocode}
%
%%%%% WŁASNE DEFINICJE I POLECENIA
%
\theoremstyle{definition} \newtheorem{definition}{Definition}[chapter]
\theoremstyle{remark} \newtheorem{remark}[definition]{Observation}
\theoremstyle{plain} \newtheorem{theorem}[definition]{Theorem}
\theoremstyle{plain} \newtheorem{lemma}[definition]{Lemma}
% \renewcommand \qedsymbol {\ensuremath{\square}}
% ...
%%%%%

\begin{document}

%%%%% POCZĄTEK ZASADNICZEGO TEKSTU PRACY

\chapter{Introduction}
TODO: improve

Guarded Fragment was introduced in \cite{?} and has been studied since. Its satisfiability problem is $2-\mathit{EXPTIME}$ complete \cite{Gradel1997}).
In \cite{BBE, jelia2021} authors introduce a restriction of the Guarded-Fragment inspired by the Fluted-Fragment (\cite{Fluted}) called Forward Guarded Fragment (FGF) enjoying \emph{EXPTIME} complexity for the satisfiability problem and the tree-model property. 
(opcjonalne)The following algorithm deciding satisfiability is though of little practicality, as the proof makes use of the $APSpace$ class.
Later it was discovered (\cite{?}) that FGF reduces to modal logic (\cite{modal logic}) with a non-deterministic polynomial time reduction(???). 
\par In \cite{resolution-gf} authors show how to decide GF with resolution. Here we adapt their work for FGF.
We rely on their proof for completeness, but derive new complexity bound. We also provide the implementation.

\chapter{The Forward Guarded Fragment}

The Forward Guarded Fragment is a restriction of the Guarded Fragment to formulas where variables of atomic formulas are infixes of the series of quantified variables. 
\begin{definition}
Let us define the Guarded Fragment (GF) as the smallest subset of first order logic satisfying:
\begin{enumerate}
    \item Atomic formulas without function symbols are in GF
    \item GF is closed under use of logical connectives
    \item If $\phi(\bar{x}, \bar{y}) \in \mathit{GF}$ where $\bar{x}, \bar{y}$ are all the free variables of $\phi$ and formula $\alpha(\bar{x},\bar{y})$ is an atom
then also $\exists_{\bar{x}} \alpha(\bar{x}, \bar{y}) \land \phi(\bar{x}, \bar{y}) \in \mathit{GF}$ and $\forall_{\bar{x}} \alpha(\bar{x}, \bar{y}) \rightarrow \phi(\bar{x}, \bar{y}) \in \mathit{GF}$
\end{enumerate}
\end{definition}
\begin{definition}
To define the Forward Fragment let us first fix a sequence of variables: $x_1, x_2, \dots$. The Forward Fragment (FF) is then the smallest subset of first order logic satisfying:
\begin{enumerate}
    \item Atomic formulas of form $R(x_i, x_{i+1},\dots,x_{j})$, that is atoms whose variables in order are infixes (without gaps) of the above sequence, are in FF
    \item FF is closed under use of logical connectives
    \item If $\phi(x_1,\dots, x_{n}) \in \mathit{FF}$ then also $\exists_{\bar{x_{n}}} \phi(x_1,\dots, x_{n}) \in FF$ and $\forall_{x_n} \phi(x_1,\dots, x_{n}) \in \mathit{FF}$
\end{enumerate}
\end{definition}
So we use the fixed sequence of variables as the order of quantification. The literals in a formula use infixes of the quantified sequence.
\begin{definition}
The Forward Guarded Fragment (FGF) is the intersection of the Guarded Fragment and the Forward Fragment.
\end{definition}

\section{Problem definition}

The goal is to describe a resolution based procedure deciding the Forward Guarded Fragment.
The algorithm should take FGF sentence as an input and return true or false based on whether the sentence is satisfiable.

\begin{definition}
A first order logic formula is called \emph{satisfiable} if there exist a model where the formula gets interpreted as true.
\end{definition}

Resolution procedure calculates a set of clauses following from the initial formula.
An empty clause -- a contradiction -- gets derived for unsatisfiable formulas. 
Then a syntactic proof of the negation of the input formula could be derived from the resolution process.
The process for arbitrary first order formulas may not terminate.
For the Guarded Fragment though, it has been shown in \cite{resolution GF}, that the production of new clauses saturates
and when it does, the model where formula holds can be derived from the saturation.
For FGF, which is of our interest, the saturation is achieved quicker.
We will need a flavour of resolution called ordered resolution, which restricts which inferences are allowed.

\chapter{Resolution procedure}

We will use an instance of the procedure from \cite{resolution gf}. 
As FGF is a subset of GF, it can be applied to FGF as well.
Stronger restriction on the input formula guarantees faster termination.

\section{Procedure overview}

First a formula needs to be translated to CNF form, that is a conjunction of clauses of literals. 
We represent a CNF formula as a set of clauses. The transformed formula is a starting point to the resolution procedure.
Resolution iterates on the set of clauses, inspecting pairs of clauses for possible inferences.
There is two ways to make an inference: resolution and factoring, described in section \textit{Inference Rules}.
When an inference is made a new clause gets added to the set.
Thus in the process the set grows containing increasingly more clauses following from the initial formula.
When an empty clause gets derived, it proves unsatisfiability of the initial sentence.
Otherwise the process stops after no new clauses can be derived. If the resulting 
clause set contains no empty clauses, the initial formula is satisfiable.

\section{Clausification}

We will describe a sequence of transformations going from formulas in FGF to formulas in CNF.
The transformations are standard for first order logic and preserve satisfiability.
\begin{enumerate}
    \item First NNF is the transformation to negation normal form (NNF). 
    It works by recursively pushing negation signs towards the atoms.
    It does that by repeatedly applying rewrites following from the De Morgans' laws. 
    See \cite{resolution book} Section 2.2 for description.
    \item \emph{Struct\textsubscript{{$\forall$}}} is the transformation applied to formulas in NNF returning a set of formulas of form $\forall_{\bar{x}}\phi(\bar{x})$ where $\phi$ is already without universal quantifiers.
    The conjunction of formulas from the resulting set is equisatisfiable with the initial formula.
    It works, by repeatedly replacing subformulas of form $\forall_{\bar{y}} \psi(\bar{x},\bar{y})$ 
    by fresh atoms $a(\bar{x})$ and adding a defining formula $\forall_{\bar{x}} a(\bar{x}) \rightarrow \forall_{\bar{y}} \psi(\bar{x},\bar{y})$.
    \item \emph{Skolemization} is the transformation removing all existential quantifiers and replacing the respective quantified variables with fresh function terms.
    Namely when a quantifier $\exists_{x_i}$ gets removed, we substitute a new function term $x_i^{\alpha}(\bar{x_{1..j}})$ for the variable it bound, 
    where $\alpha$ is a unique identifier for the given quantifier and $\bar{x_{1..j}}$ is the sequence of universally quantified variables in the scope.
    We apply it to every formula in the set resulting from Struct\textsubscript{{$\forall$}} transformation.
    % Note that then every existential quantifier is in the scope of the same universal quantifiers.
    \item Finally \emph{clausification} yields a formula in CNF by treating the formulas under universal quantifiers as propositional logic sentences.
    We represent it as a set of clauses and make the universal quantification implicit for all the free variables. 
\end{enumerate}

For the definitions of the transformations 2, 3 and 4 check the definitions 2.6, 2.7 and 2.8 respectively in \cite{resolution GF}.

\begin{definition}
Let CNF be a function from the set of FGF formulas to the set of conjunctive sets of clauses obtained by sequencing the above transformations.
\end{definition}

\section{Inference rules}\label{section:inference}

\subsection{Order}

First we recall the order on literals from \cite{resolution gf}.
By $\mathit{Vardepth}$ of a term we denote the maximal depth at which variable occurs in the term, that is:
\begin{enumerate}
    \item $\mathit{Vardepth}(A)=-1$ if $A$ is ground
    \item $\mathit{Vardepth}(A)=0$ if $A$ is a variable
    \item $\mathit{Vardepth}(f(t_1,\dots, t_i))=1+\max\{\mathit{Vardepth}(t_1), \dots, \mathit{Vardepth}(t_i)\}$ if $A$ is a term
\end{enumerate}
By $\mathit{Vardepth}$ of a literal $R(t_1, \dots, t_i)$ we denote $1+\max\{\mathit{Vardepth}(t_1), \dots, \mathit{Vardepth}(t_i)\}$.
By $\mathit{Var}$ we denote the set of variables appearing in a literal or a term.
\begin{definition}\label{def:order}
Let us define the following order $\sqsubset$ on literals.
\begin{enumerate}
    \item $A \sqsubset B$ if $Vardepth(A) < Vardepth(B)$, or
    \item $A \sqsubset B$ if $\mathit{Var}(A) \subseteq Var(B)$.
\end{enumerate}
\end{definition}
Eventhough not an order on the set of arbitrary literals it is an order among literals from a single guarded clause as taking part in the resolution.
For a proof see \cite{resolution gf}.

We also recall the following lemma from \cite{resolution GF}.
\begin{lemma}\label{lem:guarded}
Every guarded clause c has a $\sqsubset$-maximal literal, and every maximal literal
of c contains all variables of c.
\end{lemma}
For proof see Lemma 3.7 in \cite{resolution GF}.

\subsection{Normalization}
We do not want to leave choice for most general unifier at the unification step of resolution.
For that we will need normalizing renaming.

\begin{definition}
We call the following renaming a \emph{normalization} of a literal:
\begin{enumerate}
    \item Order variable occurences lexicographically on $(-\mathit{depth}, \mathit{index})$ 
    where $\mathit{depth}$ is the depth at which a position of the variable occurs and $\mathit{index}$ is a position from left where the variable occurs when literal is written in standard notation.
    \item Greedly assign variables $x_1, x_2, \dots$ in order
\end{enumerate}
\end{definition}

It is convenient to define normalization in this way as it is well-defined on all first order logic formulas.
In practice the terms produced in resolution are variables or Skolem terms and the Skolem terms contain every variable of the literal.
Therefore, if Skolem terms are present, then normalization assigns names in order 
starting from $x_1$ to the variables of the Skolem terms which names all variables already.
When literal has no Skolem terms, then normalization simply assigns variables in the order of appearence when written.
For example these two literals are normalized: $$R(x_1,x_2), Q(x_3, x_2, f(x_1,x_2, x_3))$$

\subsection{Most general unifier}
\begin{definition}\label{def:mgu}
A \emph{unifier} of two literals/terms is a substitution which applied to the literals/terms makes them syntactically equal. 
A \emph{most general unifier} (mgu) is a unifier $\sigma$, such that for every unifier $\tau$ there is a substitution $\epsilon$ so that $\tau=\epsilon\circ\sigma$.
\end{definition}
Any most general unifier composed with a renaming substitution is also a most general unifier.
We will write $A\theta$ to signify the result of applying subsitution $\theta$ to literal $A$
and $c\theta$ to signify the result of applying substitution $\theta$ to every literal of a clause $c$.

In the resolution algorithm from \cite{resolution gf} we will additionaly specify which mgu is used at the unification step,
whereas the initial authors left it unspecified.
We are allowed to do that as mgu's for a fixed unification problem differ by renamings only,
but renamings influence neither the $\sqsubset$-order nor the remaining valid inferences.
The lemma below specifies the mgu.
\begin{lemma}
Let $A_1$, $A_2$ be two literals with an mgu $\theta$.
Then there exist an mgu $\theta'$ such that $A_1\theta'=A_2\theta'$ is $normalized$.
\end{lemma}
\begin{proof}
Let $\theta'$ be the substitution obtained by applying $\theta$ first and then the normalizing renaming of the literal $A_1\theta$.
Substitution $\theta'$ is an mgu as it is an mgu composed with a renaming.
\end{proof}

We can now describe the rules for infering new clauses.

\subsection{Factoring}

\begin{definition}\label{def:factoring}
Let $c_1=\{A_1, A_2\} \cup R$ be a clause,
such that $A_1$ is maximal in $c_1$ with respect to $\sqsubset$-order from Definition~\ref{def:order}. (TODO: verify number didnt change)
and $A_1, A_2$ have a most general unifier $\theta$ such that $A_1\theta=A_2\theta$ is $normalized$.
Then the clause $\{A_1\theta\}\cup \{r\theta : r \in R\}$ is called $\sqsubset$-ordered factor of $c_1$.
\end{definition}

\subsection{Resolution}

\begin{definition}\label{def:resolution}
Let $c_1=\{A_1\} \cup R_1$ and $c_2=\{\lnot A_2\} \cup R_1$ be two clauses,
such that both $A_1$ and $\lnot A_2$ are maximal in their respective clauses with respect to $\sqsubset$-order,
$\epsilon$ be a variable renaming such that $A_1\epsilon$ doesn't share variables with $A_2$
and $A_1\epsilon$ and $A_2$ have a most general unifier $\theta$ such that $A_1\epsilon\theta=A_2\theta$ is $normalized$.
Then the clause $R_1\epsilon\theta \cup R_2\theta\}$ is called $\sqsubset$-ordered resolvent of $c_1$ and $c_2$.
\end{definition}

\section{Full algorithm}

\begin{algorithm}
\begin{algorithmic}
\Procedure{SAT}{$\phi$}
\State $C \gets CNF(\phi)$
\State $continue \gets True$
\While{continue}
\State $continue \gets False$
\For{$c_1, c_2 \in C \times C$}
\If{$c_1, c_2$ resolve into $c$}
    \State $C \gets C \cup \{c\}$
    \State $continue \gets True$
\EndIf
\If{$c_1$ factors into $c$}
    \State $C \gets C \cup \{c\}$
    \State $continue \gets True$
\EndIf
\EndFor
\EndWhile
\State
\Return $\{\} \stackrel{?}{\in} C$
\EndProcedure
\end{algorithmic}
\end{algorithm}

The above algorithm is an instance of the resolution procedure from \cite{resolution GF}.

\chapter{Completeness}

The resolution algorithm derives a set of clauses and answers the satisfiability question 
based on whether the set contains the empty clause. The algorithm is complete, because 
the empty clause is guaranteed to be derived for unsatisfiable sentences.
This is true about unrestricted resolution and arbitrary first order logic sentences, 
but also about the Guarded-Fragment and ordered resolution described above as was shown in \cite{resolution gf}.

\begin{theorem}
Algorithm SAT decides satisfiability of FGF sentences.
\end{theorem}

\begin{proof}
From Theorem 3.20 of \cite{resolution gf} we know that SAT decides GF sentences and FGF is a subset of GF.
\end{proof}

\chapter{Complexity}

The stronger restriction on FGF formulas compared to GF formulas gives a restriction on produced clauses which we call $forwardness$.

\section{Forwardness}

\begin{definition}\label{def:notation}
We will write $\bar{x}_{i..j}$ for the gap-free sequence of variables $x_i, x_{i+1}, \dots, x_j$ and 
$\bar{x}^{\bar{\alpha}}_{j..k}(\bar{x}_{1..i})$ for the gap-free sequence of Skolem terms
$$x^{\alpha_{j}}_{j}(\bar{x}_{1..i}), x^{\alpha_{j+1}}_{j+1}(\bar{x}_{1..i}), \dots, x^{\alpha_k}_k(\bar{x}_{1..i})$$.
\end{definition}

\begin{definition}\label{def:forward}
We call a literal $forward$ if it is of form
$$(\lnot)R(\bar{x}_{i..j}, \bar{x}^{\bar{\alpha}}_{{j+1..k}}(\bar{x}_{1..j}))$$
for some relational symbol $R$
and sequence $\bar{x}^{\bar{\alpha}}_{{j+1..k}}=x^{\alpha_{j+1}}_{j+1}, \dots, x^{\alpha_{k}}_{k}$ of 
Skolem function symbols assigned in skolemization to a sequence $\exists_{j+1}, \dots, \exists_{k}$ of quantifiers 
such that quantifier $\exists_{k}$ was in scope of quantifiers $\exists_{j+1}, \dots, \exists_{k-1}$.
This includes ground literals. Variable or Skolem term sequences may be empty.
\end{definition}

The condition on Skolem symbols says that a sequence $\bar{\alpha}$ is a
sequence of identifiers assigned at skolemization step to existential quantifiers
at some path from root of the formula to the subformula of the quantifier $\exists_k$.
% We will refer to this condition by saying that $\bar{alpha}$ is an existential path.

We will call a clause $forward$ if its literals are $forward$.
% $Forward$ clauses are already normalized, because variables $\bar{x}_{1..j}$ of Skolem terms are numbered in order starting from 1
% and Skolem terms contain all variables

\begin{lemma}\label{lem:normalization}
Let $A$ be a forward literal, $\sigma$ be its normalization and $c$ be a forward clause
such that $B\sqsubset A$ for all literals $B \in c$.
Then $c\sigma$ is forward.
\end{lemma}
\begin{proof}
Take $A$, $\sigma$ and $c$ as above. Literal $A$ is greater in the $\sqsubset$ order from literas in $c$, therefore it contains all the variables of $c$.

If $c$ is ground, then $c\sigma=\sigma$ and therefore $c\sigma$ is forward.
Let's now assume that $c$ is not ground and therefore $A$ is also not ground.

If $A$ contains a Skolem term then it is normalized, 
because the Skolem term contains all variables of $A$ and the variables are ordered starting from 1.
In this case $c\sigma=c$ so $c\sigma$ is forward.

In the other case the literal $A$ is of form $R(\bar{x}_{i..j})$ for some relation symbol $R$ and indices $i$, $j$.
Then $A$ after normalization is the literal $R(\bar{x}_{1..j-i+1})$ and normalization is the renaming $x_k\mapsto x_{k-i+1}$ for $k=i,\dots,j$.
Recall that here $1..j-i+1$ denotes the interval from $1$ to $j-i+1$.
By the assumptions that literal $A$ is greater in the $\sqsubset$ order than all the literals of the clause $c$, 
it has no smaller $\mathit{Vardepth}$ than the literals of $c$ and therefore in $c$ there are no Skolem terms.
Literals from $c$ are forward, use variables $x_i, \dots, x_j$ and do not contain Skolem terms.
Therefore the arguments of atoms from $c$ are infixes of the tuple $\bar{x}_{i..j}$ and 
the arguments of atoms from $c\sigma$ are infixes of the tuple $\bar{x}_{1..j-i+1}$.
So $c\sigma$ is forward.
\end{proof}


The below and the next lemmas guarantee that only forward clauses get derived in the resolution process.
\begin{lemma}
If $\phi \in \mathit{FGF}$ then $\mathit{CNF}(\phi)$ contains only $forward$ clauses.
\end{lemma}

\begin{proof}
Take $\phi \in \mathit{FGF}$. Atoms of $\phi$ are of form $R(x_{i..j})$.
Let us consider the steps of CNF.TODO

First NNF
\end{proof}

% \begin{lemma}\ \\
\begin{lemma}
\begin{enumerate}
    \item If $c_1$, $c_2$ are $forward$ clauses and $c$ is ordered resolvent of $c_1$ and $c_2$, then $c$ is forward.
    \item If $c_1$ is $forward$ clause and $c$ is a factor of $c_1$, then $c$ is $forward$.  
\end{enumerate}
\end{lemma}

\begin{proof}
\par We consider resolution first. Let $c_1$, $c_2$ be $forward$ clauses and $c$ be their ordered resolvent.
Let $A\in c_1$ and $B\in c_2$ be the literals resolved upon.
Without loss of generality let $A$ be the positive literal.
Then $A=R(\bar{x}_{k..l}, \bar{x}^{\bar{\alpha}}_{{l+1..m}}(\bar{x}_{1..l}))$
and $B=\lnot R(\bar{x}_{k+s..o}, \bar{x}^{\bar{\alpha'}}_{{o+1..m+s}}(\bar{x}_{1..o}))$ for some $k,l,m,o\in \mathbb{N}$, $s\in \mathbb{Z}$
and sequences $\bar{\alpha}$, $\bar{\alpha'}$ as in the Definition~\ref{def:forward}. 
% Recall that here $k+s..o$ denotes the interval from $k+s$ to $o$.
We denoted by $k, m$ the interval of indices appearing in $A$, by $s$ the shift compared to $B$.
Then $l, o$ mark indices where the sequence of variables turns into sequence of Skolem terms in $A$ and $B$ respectively.
Let us also assume that in $A$ the prefix of variables $\bar{x}_{i..j}$
is no shorter than in $B$, that is $l-k\geq o-(k+s)$. Two assumptions can be both made without loss of generality as
the polarity of literals doesn't impact the unifier.
To calculate the most general unifier let's first rename variables of $A$: $x_i \mapsto y_i$.
The unification problem is:
\begin{align*}
&R(&y_k&, \dots, &y_{o-s}&,&y_{o-s+1}&,\dots,                                  &y_l&,                                     &x^{\alpha_{l+1}}_{l+1}(\bar{y}_{1..l})&, \dots,      &x^{\alpha_{m}}_{m}(\bar{y}_{1..l})&) \\
    \doteq \\
\lnot &R(&x_{k+s}&, \dots, &x_{o}&,&x^{\alpha'_{o+1}}_{o+1}(\bar{x}_{1..o})&, \dots, &x^{\alpha'_{l+s}}_{l+s}(\bar{x}_{1..o})&, &x^{\alpha'_{l+s+1}}_{l+s+1}(\bar{x}_{1..o})&, \dots, &x^{\alpha'_{m+s}}_{m+s}(\bar{x}_{1..o})&)
\end{align*}
Comparing the terms we get 3 types of equations:
\begin{enumerate}
    \item $y_i\doteq x_{i+s}$ for $i=k,\dots, o-s$
    \item $y_i\doteq x^{\alpha'_{i+s}}_{i+s}(\bar{x}_{1..o})$ for $i=o-s+1,\dots, l$
    \item $x_i^{\alpha_{i}}(\bar{y}_{1..l})\doteq x^{\alpha'_{i+s}}_{i+s}(\bar{x}_{1..o})$ for $s=l+1,\dots, m$
\end{enumerate}
Every mgu of renamed $A$ and $B$ satisfies the equations.

If there are any equations of type 3 and there exists a solution, then
necessarily $x^{\alpha_m}_m$ and $x^{\alpha'_{m+s}}_{m+s}$ are the same function symbols, so $s=0$.
Also $\bar{y}_{1..l}=\bar{x}_{1..o}$, so $o=l$ and $y_i=x_i$ for $i=1,\dots,l$.
It follows that the clauses $c_1$, $c_2$ are already unified with the identity unification.
Literals $A$ and $B$ are also normalized, because Skolem terms contain all variables and the variables are ordered starting from 1.
Therefore every literal in the clause $c$ comes directly from one of $c_1$, $c_2$, so $c$ is forward.

TODO: sigma doesnt necessarily normalize
Let us consider the other case. Then there are no equations of type 3 and $A$ doesn't contain any function terms.
In this case the unification has an easy solution. % We can then solve the unification problem using variables from $c_2$ in the result.
We will define the unifying substitution $\sigma : \{y_k,\dots,y_l, x_1, \dots, x_o\} \rightarrow \{x_1, \dots, x_o\}$.
Let $\sigma$ be the identity substitution on variables $x_1$ to $x_o$. Therefore $B\sigma=B$.
Equations of type 1 and 2 define the substitution on variables $y_k$ to $y_l$:
\begin{itemize}
    \item $y_i \leftarrow x_{i+s}$ for $i=k,\dots, o-s$
    \item $y_i \leftarrow x^{\alpha'_{i+s}}_{i+s}(\bar{x}_{1..o})$ for $i=o-s+1,\dots, l$
\end{itemize}
The substitution $\sigma$ is trivially a unifier of renamed $A$ and $B$, as
$\sigma$ unifies all pairs of relational symbol arguments at matching indices.
The unifier $\sigma$ is the most general unifier, because every unifier more general
has to assign a variable instead of a function term to one of the variables $y_{o-s+1}$ to $y_l$, 
thus invalidating the corresponding equation.
% We need the unifier that leaves literals normalized. 
% If $B$ contains a Skolem term, then $B\sigma=B$ is normalized, because the Skolem term contains all variables of $B$ in the $x_1,x_2\dots$ order.
% If $B$ doesn't have Skolem terms, then $B=\lnot R(\bar{x}_{k+s..o})$ and the normalization renames it to $B=\lnot R(\bar{x_{1..o-k-s+1}})$.
% So let $\sigma'$ be the normalizing mgu:
% \begin{equation}
% \sigma'=\begin{cases}
%     \sigma &\text{$B$ contains a Skolem term} \\
%     (\lambda x_i \rightarrow x_{i-k-s+1})\circ\sigma &\text{$B$ doesn't contain Skolem terms}
% \end{cases}
% \end{equation}
The substitution $\sigma'$ is the unifier used in the resolution procedure, because $B$ is $normalized$ after applying $\sigma$.
Therefore every literal in $c$ is either a literal of $c_2$ and therefore $forward$, or it is 
literal of $c_1$ after applying substitution $\sigma$.
$A$ is a maximal literal in $c_1$, so every literal in $c_1$ 
\begin{itemize}
    \item doesn't contain function terms
    \item contains only variables from $A$ by lemma (TODO:lemma number)
\end{itemize}
Therefore every literal in $c_1$ is of form $(\lnot)R(\bar{y}_{i..j})$ for some infix $\bar{y}_{i..j}$ of $\bar{y}_{k..l}$.
After substitution every literal is of form $(\lnot)R(\bar{t})$ for some infix $\bar{t}$ of $\bar{x}_{k+s..o}, \bar{x}^{\bar{\alpha'}}_{{o+1..m+s}}(\bar{x}_{1..o})$, therefore is $forward$.

\par Let us now consider factoring. Let $c_1$ be a $forward$ clause and $c$ be its factor.
Let $A_1$, $A_2$ be the literals participating in factoring and $A_1$ be the maximal one.
Literal $A_1$ contains all variables of $A_2$ by Lemma~\ref{lem:guarded} and both literals are $forward$. 
It follows that the literals are indentical. $A_1$ is already $normalized$, because it is $forward$.
The factoring substitution is then identity substitution and $c\subset c_1$. So $c$ is forward.
\end{proof}

\begin{lemma}
Let $\phi$ be an FGF formula with $l$ existential quantifiers and 
$A$ be the set of Skolem function symbols in $CNF(\phi)$.
Let $m$ be the number of relational symbols in $\phi$, $a$ be the maximal arity
of relational symbols and $n$ be the number of variables in $\phi$. 
Then there is at most $2\cdot m \cdot n^2 \cdot l$ forward literals using relations from $\phi$ and function symbols from $A$.
\end{lemma}
\begin{proof}
A forward literal $(\lnot)R(\bar{x}_{i..j}, \bar{x}^{\bar{\alpha}}_{{j+1..k}}(\bar{x}_{1..j}))$
begins with a possibly negated relational symbol giving $2\cdot m$ options. 
Then one of $n^2$ infixes of sequence $x_1, \dots, x_n$ follows.
Then a sequence of Skolem terms follows.
The sequence ends with some Skolem term $x^{\alpha_k}_k(\bar{x}_{1..j})$ and that term 
already defines the sequence of Skolem terms $\bar{x}^{\bar{\alpha_{j+1..k-1}}}_{{j+1..k-1}}(\bar{x}_{1..j}))$,
because in $\phi$ the quantifier identified by $\alpha_k$ is in the scope of exactly one sequence of quantifiers identified by some $\alpha_{j+1},\dots, \alpha_{k-1}$.
Therefore there is at most $2\cdot m \cdot n^2 \cdot l$ forward literals from the lemma.
\end{proof}

\section{Procedure complexity}

\begin{theorem}
Procedure $SAT$ works in exponential time with respect to the length of the input formula. 
\end{theorem}

\begin{proof}
The complexity is made up by the complexity of the CNF transformation plus the complexity of the following resolution process.
Let $n$ be the length of the input formula.
Let us consider clausification first:
\begin{enumerate}
    \item \emph{NNF} works in linear time with respect to the length of the formula and increases the size of the formula by a constant factor.
    \item The output of \emph{Struct\textsubscript{{$\forall$}}} transformation is a set of formulas. There is at most as many formulas as there is 
    subformulas of the input and they are no bigger than the input, so the output is at most of quadratic size compared to the input.
    Therefore \emph{Struct\textsubscript{{$\forall$}}} works in at most quadratic time.
    \item The skolemization works in linear time and increases the formula by a constant factor.
    \item Finally, clausification may produce exponentially many polynomially sized clauses.
\end{enumerate}
Clearly, the complexity of CNF translation is no bigger than exponential.
Then the resolution process starts. The process ends once all possible clauses get derived.
From lemmas 6.3 and 6.4 we know that all the derived terms will be $forward$.
From lemma 6.5 we know that there is at most $2\cdot n^4$ $forward$ terms.
Therefore there is at most $2^{2\cdot n^4}$ $forward$ clauses. 
Algorithm SAT has to inspect every possible pair of clauses to derive new clause or terminate.
Inspecting a pair of clauses takes polynomial time with respect to $n$, because 
a clause size is no bigger than $2\cdot n^4$ and every literal of the clause is at most of quadratic size wrt. $n$.
Therefore the resolution process also terminates after exponentially many steps.
We conclude that SAT works in exponential time.
\end{proof}

\chapter{Implementation}

\section{User guide}

TODO

%%%%% BIBLIOGRAFIA

% \begin{thebibliography}{1}
%     Grädel, E., Kolaitis, P., Vardi, M., 1997. On the decision problem for two-variable first-order logic.
% Bull. Symb. Logic 3, 53–69
% \end{thebibliography}

\end{document}
